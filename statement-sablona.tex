% Soubory musí být v kódování, které je nastaveno v příkazu \usepackage[...]{inputenc}

\documentclass[%        Základní nastavení
%  draft,    				  % Testovací překlad
  14pt,       				% Velikost základního písma je 12 bodů
  a4paper,    				% Formát papíru je A4
  %oneside,      			% Jednostranný tisk
	twoside,      			% Dvoustranný tisk (kapitoly a další důležité části tedy začínají na lichých stranách)
	unicode,						% Záložky a metainformace ve výsledném  PDF budou v kódování unicode
]{extreport}				  % Dokument třídy 'extreport', potreba kvuli velikosti pisma

\usepackage[utf8]		  %	Kódování zdrojových souborů je UTF-8
	{inputenc}					% Balíček pro nastavení kódování zdrojových souborů

\usepackage[				% Nastavení geometrie stránky
	bindingoffset=0mm,		% Hřbet pro vazbu
	hmargin={21mm,21mm},	% Vnitřní a vnější okraj
	vmargin={21mm,28mm},	% Horní a dolní okraj
	footskip=14mm,			  % Velikost zápatí
	nohead,					      % Bez záhlaví
	marginparsep=0mm,		  % Vzdálenost marginálií
	marginparwidth=0mm,	  % Šířka marginálií
]{geometry}

%\usepackage{sectsty}
	%přetypuje nadpisy všech úrovní na bezpatkové, kromě \chapter, která je přenastavena zvlášť v thesis.sty
	%\allsectionsfont{\sffamily}

\usepackage{graphicx} % Balíček 'graphicx' pro vkládání obrázků
											% Nutné pro vložení logotypů školy a fakulty

\usepackage[          % Balíček 'acronym' pro sazby zkratek a symbolů
	nohyperlinks				% Nebudou tvořeny hypertextové odkazy do seznamu zkratek
]{acronym}						
											% Nutné pro použití prostředí 'acronym' balíčku 'thesis'

\usepackage[
	breaklinks=true,		% Hypertextové odkazy mohou obsahovat zalomení řádku
	hypertexnames=false % Názvy hypertext. odkazů budou tvořeny nezávisle na názvech TeXu
  hidelinks,
  colorlinks,
  allcolors=black
]{hyperref}						% Balíček 'hyperref' pro sazbu hypertextových odkazů
											% Nutné pro použití příkazu 'pdfsettings' balíčku 'thesis'

\usepackage{pdfpages} % Balíček umožňující vkládat stránky z PDF souborů
                      % Nutné při vkládání titulních listů a zadání přímo
                      % ve formátu PDF z informačního systému

\usepackage{enumitem} % Balíček pro nastavení mezerování v odrážkách
  \setlist{topsep=0pt,partopsep=0pt,noitemsep} % konkrétní nastavení

\usepackage{cmap} 		% Balíček cmap zajišťuje, že PDF vytvořené `pdflatexem' je
											% plně "prohledávatelné" a "kopírovatelné"

%\usepackage{upgreek}	% Balíček pro sazbu stojatých řeckých písmem
											%% např. stojaté pí: \uppi
											%% např. stojaté mí: \upmu (použitelné třeba v mikrometrech)
											%% pozor, grafická nekompatibilita s fonty typu Computer Modern!
                      
%\usepackage{amsmath} %balíček pro sabu náročnější matematiky                 

\usepackage{dirtree}	% sazba adresářové struktury
                      % vhodné pro prezentaci obsahu elektronické přílohy (např. CD)

\usepackage[formats]{listings}	% Balíček pro sazbu zdrojových textů
\lstset{              % nastavení
%	Definice jazyka použitého ve výpisech
%    language=[LaTeX]{TeX},	% LaTeX
%	language={Matlab},		% Matlab
	language={C},           % jazyk C
    basicstyle=\ttfamily,	% definice základního stylu písma
    tabsize=2,			% definice velikosti tabulátoru
    inputencoding=utf8,         % pro soubory uložené v kódování UTF-8
		columns=fixed,  %fixed nebo flexible,
		fontadjust=true %licovani sloupcu
    extendedchars=true,
    literate=%  definice symbolů s diakritikou
    {á}{{\'a}}1
    {č}{{\v{c}}}1
    {ď}{{\v{d}}}1
    {é}{{\'e}}1
    {ě}{{\v{e}}}1
    {í}{{\'i}}1
    {ň}{{\v{n}}}1
    {ó}{{\'o}}1
    {ř}{{\v{r}}}1
    {š}{{\v{s}}}1
    {ť}{{\v{t}}}1
    {ú}{{\'u}}1
    {ů}{{\r{u}}}1
    {ý}{{\'y}}1
    {ž}{{\v{z}}}1
    {Á}{{\'A}}1
    {Č}{{\v{C}}}1
    {Ď}{{\v{D}}}1
    {É}{{\'E}}1
    {Ě}{{\v{E}}}1
    {Í}{{\'I}}1
    {Ň}{{\v{N}}}1
    {Ó}{{\'O}}1
    {Ř}{{\v{R}}}1
    {Š}{{\v{S}}}1
    {Ť}{{\v{T}}}1
    {Ú}{{\'U}}1
    {Ů}{{\r{U}}}1
    {Ý}{{\'Y}}1
    {Ž}{{\v{Z}}}1
}

%%%%%%%%%%%%%%%%%%%%%%%%%%%%%%%%%%%%%%%%%%%%%%%%%%%%%%%%%%%%%%%%%
%%%%%%      Definice informací o dokumentu             %%%%%%%%%%
%%%%%%%%%%%%%%%%%%%%%%%%%%%%%%%%%%%%%%%%%%%%%%%%%%%%%%%%%%%%%%%%%

\input{nastaveni}  % do tohoto souboru doplňte údaje o sobě, druhu práce, názvu...

%%%%%%%%%%%%%%%%%%%%%%%%%%%%%%%%%%%%%%%%%%%%%%%%%%%%%%%%%%%%%%%%%%%%%%%%

%%%%%%%%%%%%%%%%%%%%%%%%%%%%%%%%%%%%%%%%%%%%%%%%%%%%%%%%%%%%%%%%%%%%%%%%
%%%%%%     Nastavení polí ve Vlastnostech dokumentu PDF      %%%%%%%%%%%
%%%%%%%%%%%%%%%%%%%%%%%%%%%%%%%%%%%%%%%%%%%%%%%%%%%%%%%%%%%%%%%%%%%%%%%%
%% Při načteném balíčku 'hyperref' lze použít příkaz '\pdfsettings':
\pdfsettings
%  Nastavení polí je možné provést také ručně příkazem:
%\hypersetup{
%  pdftitle={Název studentské práce},    	% Pole 'Document Title'
%  pdfauthor={Autor studenstké práce},   	% Pole 'Author'
%  pdfsubject={Typ práce}, 						  	% Pole 'Subject'
%  pdfkeywords={Klíčová slova}           	% Pole 'Keywords'
%}
%%%%%%%%%%%%%%%%%%%%%%%%%%%%%%%%%%%%%%%%%%%%%%%%%%%%%%%%%%%%%%%%%%%%%%%

\pdfmapfile{=vafle.map}

%%%%%%%%%%%%%%%%%%%%%%%%%%%%%%%%%%%%%%%%%%%%%%%%%%%%%%%%%%%%%%%%%%%%%%%
%%%%%%%%%%%       Začátek dokumentu               %%%%%%%%%%%%%%%%%%%%%
%%%%%%%%%%%%%%%%%%%%%%%%%%%%%%%%%%%%%%%%%%%%%%%%%%%%%%%%%%%%%%%%%%%%%%%
\begin{document}
\pagestyle{empty} %vypnutí číslování stránek

%% Vložení desek 
%\includepdf[pages=1]%  buďto generovaných informačním systémem
%  {pdf/student-desky}% název souboru nesmí obsahovat mezery!
%% NEBO vytvoření desek z balíčku
%\makecover
%%
%\oddpage % při dvojstranném tisku přidá prázdnou stránku
% kazdopádně ale:
\setcounter{page}{1} %resetovaní čítače stránek -- desky do číslování nezahrnujeme

%% Vložení titulního listu
%\includepdf[pages=1]%    buďto generovaného informačním systémem
%  {pdf/student-titulka}% název souboru nesmí obsahovat mezery!
%% NEBO vytvoření titulní stránky z balíčku
\maketitle
%%
\oddpage  % při dvojstranném tisku se přidá prázdná stránka
   
%% Vložení zadání
%\includepdf[pages=1]%   buďto generovaného informačním systémem
%  {pdf/student-zadani}% název souboru nesmí obsahovat mezery!
%% NEBO lze vytvořit prázdný list příkazem ze šablony
%\patternpage{}%
%	{\sffamily\Huge\centering ZDE VLOŽIT LIST ZADÁNÍ}%
%	{\sffamily\centering Z~důvodu správného číslování stránek}
%%
%\oddpage% při dvojstranném tisku se přidá prázdná stránka

%% Vysázení stránky s abstraktem
%\makeabstract

%% Vysázení druhé strany tezí - klíčová slova
\statementscndpg

% Vysázení stránky s rozšířeným abstraktem
% (pokud píšete práci v češtině či slovenštině, vložení rozšířeného abstraktu zrušte;
%  pro semestrální projekt také není potřeba rozšířený abstrakt uvádět)
%\input{text/rozsireny_abstrakt}

%%% Vysázení citace práce
%\makecitation

%%% Vysázení prohlášení o samostatnosti
%\makedeclaration

%%% Vysázení poděkování
%\makeacknowledgement

%%% Vysázení obsahu
\setcounter{tocdepth}{1}
\tableofcontents

%%% Vysázení seznamu obrázků
% (vynechejte, pokud máte dva nebo méně obrázků)
%\listoffigures

%%% Vysázení seznamu tabulek
% (vynechejte, pokud máte dvě nebo méně tabulek)
%\listoftables

%%% Vysázení seznamu výpisů kódu
% (vynechejte, pokud máte dva nebo méně výpisů)
%\lstlistoflistings

\cleardoublepage\pagestyle{plain}   % zapnutí číslování stránek

\chapter*{Introduction}
Lorem ipsum dolor sit amet, consectetuer adipiscing elit. Mauris dictum facilisis augue. Morbi imperdiet, mauris ac auctor dictum, nisl ligula egestas nulla, et sollicitudin sem purus in lacus. Pellentesque habitant morbi tristique senectus et netus et malesuada fames ac turpis egestas. Proin pede metus, vulputate nec, fermentum fringilla, vehicula vitae, justo. Donec quis nibh at felis congue commodo. Duis sapien nunc, commodo et, interdum suscipit, sollicitudin et, dolor. Etiam commodo dui eget wisi. Nullam feugiat, turpis at pulvinar vulputate, erat libero tristique tellus, nec bibendum odio risus sit amet ante. Etiam neque. Duis risus. Nulla est. Nulla turpis magna, cursus sit amet, suscipit a, interdum id, felis. Mauris dictum facilisis augue. Aliquam erat volutpat. Aliquam ornare wisi eu metus. Nullam faucibus mi quis velit. In sem justo, commodo ut, suscipit at, pharetra vitae, orci.

\chapter{SOA}
Curabitur ligula sapien, pulvinar a vestibulum quis, facilisis vel sapien. Vivamus porttitor turpis ac leo. In laoreet, magna id viverra tincidunt, sem odio bibendum justo, vel imperdiet sapien wisi sed libero. Cras elementum. Mauris tincidunt sem sed arcu. Sed ac dolor sit amet purus malesuada congue. Morbi scelerisque luctus velit. Nemo enim ipsam voluptatem quia voluptas sit aspernatur aut odit aut fugit, sed quia consequuntur magni dolores eos qui ratione voluptatem sequi nesciunt. Aliquam ornare wisi eu metus. Nulla quis diam. Nunc tincidunt ante vitae massa. Etiam commodo dui eget wisi.

\chapter{Goals}
Aenean id metus id velit ullamcorper pulvinar. Nullam feugiat, turpis at pulvinar vulputate, erat libero tristique tellus, nec bibendum odio risus sit amet ante. Integer rutrum, orci vestibulum ullamcorper ultricies, lacus quam ultricies odio, vitae placerat pede sem sit amet enim. Pellentesque ipsum. Nemo enim ipsam voluptatem quia voluptas sit aspernatur aut odit aut fugit, sed quia consequuntur magni dolores eos qui ratione voluptatem sequi nesciunt. Etiam dui sem, fermentum vitae, sagittis id, malesuada in, quam. Duis sapien nunc, commodo et, interdum suscipit, sollicitudin et, dolor. Nullam feugiat, turpis at pulvinar vulputate, erat libero tristique tellus, nec bibendum odio risus sit amet ante. Praesent in mauris eu tortor porttitor accumsan. Fusce nibh. Fusce suscipit libero eget elit. Pellentesque habitant morbi tristique senectus et netus et malesuada fames ac turpis egestas.

\chapter{My work}
Aliquam ornare wisi eu metus. Praesent vitae arcu tempor neque lacinia pretium. Vivamus luctus egestas leo. Itaque earum rerum hic tenetur a sapiente delectus, ut aut reiciendis voluptatibus maiores alias consequatur aut perferendis doloribus asperiores repellat. Etiam egestas wisi a erat. Maecenas ipsum velit, consectetuer eu lobortis ut, dictum at dui. Pellentesque ipsum. Etiam neque. Morbi leo mi, nonummy eget tristique non, rhoncus non leo. Vestibulum erat nulla, ullamcorper nec, rutrum non, nonummy ac, erat. Temporibus autem quibusdam et aut officiis debitis aut rerum necessitatibus saepe eveniet ut et voluptates repudiandae sint et molestiae non recusandae. Class aptent taciti sociosqu ad litora torquent per conubia nostra, per inceptos hymenaeos. Nulla non arcu lacinia neque faucibus fringilla. Suspendisse nisl. Nunc tincidunt ante vitae massa. Phasellus rhoncus. Curabitur ligula sapien, pulvinar a vestibulum quis, facilisis vel sapien. Cras elementum. Pellentesque ipsum.

\chapter{Conclusion}
Nulla pulvinar eleifend sem. Integer lacinia. Sed ac dolor sit amet purus malesuada congue. Duis condimentum augue id magna semper rutrum. In sem justo, commodo ut, suscipit at, pharetra vitae, orci. Morbi imperdiet, mauris ac auctor dictum, nisl ligula egestas nulla, et sollicitudin sem purus in lacus. Phasellus faucibus molestie nisl. Proin mattis lacinia justo. Integer in sapien. Mauris tincidunt sem sed arcu. Aliquam erat volutpat. Etiam posuere lacus quis dolor. Fusce nibh. Integer pellentesque quam vel velit. Nulla accumsan, elit sit amet varius semper, nulla mauris mollis quam, tempor suscipit diam nulla vel leo. Duis aute irure dolor in reprehenderit in voluptate velit esse cillum dolore eu fugiat nulla pariatur. Nullam faucibus mi quis velit.

%% Vložení souboru 'text/literatura' se seznamem literatury
%\include{text/bibliography}
\bibliographystyle{IEEEtr}
\bibliography{}

\newpage
\chapter*{Curriculum vitæ}
\phantomsection
\addcontentsline{toc}{chapter}{Curriculum vitæ}

\section*{Education}

\noindent\textbf{Cybernetics, Control and Measurements}\hfill 2015-now\\
\indent Doctoral degree\\
\indent Brno University of Technology\\
\indent Faculty of Electrical Engineering and Communication\\

\noindent \textbf{Cybernetics, Control and Measurements}\hfill 2013-2015\\
\indent Master's degree\\
\indent Brno University of Technology\\
\indent Faculty of Electrical Engineering and Communication\\

\noindent\textbf{Automation and Measurement}\hfill \hfill 2009-2013\\
\indent Bachelor's degree\\
\indent Brno University of Technology\\
\indent Faculty of Electrical Engineering and Communication

\section*{Teaching and advising}
\noindent\textbf{Faculty of Electrical Engineering and Communication}\\
\indent Modeling and Simulation\\
\indent Control Theory 2\\
\indent M. Nemožný, “Název bakalářské práce”\\
\indent \indent bachelor's thesis, Vysoké učení technické v Brně, Brno, 2018.\\

\noindent\textbf{Faculty of Business and Management}\\
\indent Theory of Systems\\

\section*{Projects}
\noindent\textbf{H2020 AutoDrive} - Advancing fail-aware, fail-safe, and fail-operational electronic components, systems, and architectures for fully automated driving to make future mobility safer, affordable, and end-user acceptable.\\

\noindent\textbf{H2020 OSEM-EV} - Optimised and Systematic Energy Management in Electric Vehicles\\

\noindent\textbf{H2020 3Ccar} - Integrated Components for Complexity Control in affordable electrified cars\\

\noindent\textbf{MATERIS} - Podpora rozvoje kvalitních týmů výzkumu a vývoje v oblasti materiálových věd\\

\noindent\textbf{FEKT-S-14-2429} - The research of new control methods, measurement procedures and intelligent instruments in automation\\

\noindent\textbf{FEKT-S-17-4234} - Industry 4.0 in automation and cybernetics\\

\section*{Published papers}

\statementlstpg

\end{document}